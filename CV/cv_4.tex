%%%%%%%%%%%%%%%%%%%%%%%%%%%%%%%%%%%%%%%%%
% Medium Length Professional CV
% LaTeX Template
% Version 2.0 (8/5/13)
%
% This template has been downloaded from:
% http://www.LaTeXTemplates.com
%
% Original author:
% Trey Hunner (http://www.treyhunner.com/)
%
% Important note:
% This template requires the resume.cls file to be in the same directory as the
% .tex file. The resume.cls file provides the resume style used for structuring the
% document.
%
%%%%%%%%%%%%%%%%%%%%%%%%%%%%%%%%%%%%%%%%%

%----------------------------------------------------------------------------------------
%	PACKAGES AND OTHER DOCUMENT CONFIGURATIONS
%----------------------------------------------------------------------------------------

\documentclass{resume} % Use the custom resume.cls style
\usepackage{etaremune}
\usepackage{enumitem}
\usepackage{booktabs}
\usepackage{hyphenat}

\usepackage[left=0.75in,top=0.6in,right=0.75in,bottom=0.6in]{geometry} % Document margins


\name{Hillary A. Scannell} % Your name
\address{Lamont-Doherty Earth Observatory \\ Columbia University} % Your secondary addess (optional)
\address{61 Rte 9W \\ Palisades, NY 10964 } % Your address

\address{scannell@ldeo.columbia.edu \\ 
\href{http://www.hillaryscannell.com}{hillaryscannell.com}
} % Your phone number and email

\begin{document}

%----------------------------------------------------------------------------------------
%	EDUCATION SECTION
%----------------------------------------------------------------------------------------

\begin{rSection}{Education}

{\bf University of Washington} \hfill {Seattle, WA}  \\
Ph.D. in Oceanography  \hfill {\em 2020} 

{\bf University of Maine} \hfill {Orono, ME} \\ 
M.S. in Oceanography  \hfill {\em 2014}  \\ 
B.S. in Marine Science \hfill {\em 2013} 
\end{rSection}

%----------------------------------------------------------------------------------------
%	WORK EXPERIENCE SECTION
%----------------------------------------------------------------------------------------

\begin{rSection}{Research Appointments}

\begin{rSubsection}{Lamont-Doherty Earth Observatory at Columbia University}{ Palisades, NY}{Postdoctoral Research Scientist, Climate Data Science Lab}{\em 2020 – Present}
\end{rSubsection}

%------------------------------------------------

\begin{rSubsection}{University of Washington}{Seattle, WA}{Graduate Research Assistant, School of Oceanography}{\em 2015 – 2020}
\end{rSubsection}

%------------------------------------------------
\begin{rSubsection}{National Center for Atmospheric Research}{Boulder, CO}{Graduate Research Assistant, Computational and Information Systems Lab}{\em Spring 2020}
\end{rSubsection}

%------------------------------------------------
\begin{rSubsection}{Tableau Software}{Seattle, WA}{Research Intern}{\em Summer 2018}
\end{rSubsection}

%------------------------------------------------
\begin{rSubsection}{NOAA Pacific Marine Environmental Laboratory}{Seattle, WA}{Graduate Research Assistant, Global Tropical Moored Buoy Array Lab}{\em 2015 – 2017}
\end{rSubsection}


%------------------------------------------------
\begin{rSubsection}{University of New South Wales}{Sydney, NSW}{Visiting Research Fellow, Climate Change Research Center}{\em Summer 2014}
\end{rSubsection}

%------------------------------------------------
\begin{rSubsection}{Gulf of Maine Research Institute}{Portland, ME}{Graduate Research Assistant, Ecosystem Modeling Lab}{\em 2013 – 2015}
{Research Intern, Ecosystem Modeling Lab}{\hfill \em Summer 2013}
\end{rSubsection}

\end{rSection}

%----------------------------------------------------------------------------------------
%	Awards and Grants
%----------------------------------------------------------------------------------------

\begin{rSection}{Funded Proposals}
{\bf Scannell, H. A.}, 2020: {National Center for Atmospheric Research Advanced Study Program Graduate Fellowship}, \$6,000. \\ \indent
Thompson, L. and {\bf Scannell, H. A.}, 2020: Following the heat towards large marine ecosystems: AI tools for tracking dangerous marine heatwaves. {Leonardo DiCaprio Foundation \& Microsoft AI for Earth Innovation Grant}, \$99,889. \href{https://www.globalwildlife.org/press/winners-of-ai-for-earth-innovation-grants-poised-to-address-urgent-environmental-challenges-with-creative-use-of-technology/}{[Press Release]}, \href{https://escience.washington.edu/data-science-to-track-mhws}{[UW eScience Institute Highlight]}\\ \indent
{\bf Scannell, H. A.}, 2018–2019: {Microsoft AI for Earth Azure Compute Grant}, over \$15,000 in cloud compute credits. \\ \indent
{\bf Scannell, H. A.}, 2018: {Integral Consulting Inc. Environmental Big Data Research Award}, \$3,764. \\ \indent
{\bf Scannell, H. A.}, 2018: Impacts of El Niño-Southern Oscillation on Indian Ocean heatwaves, {National Science Foundation EAPSI/ EPSCoR Co-Funding}, \$5,070. \\
\end{rSection}
\newpage
%----------------------------------------------------------------------------------------
%	PUBLICATIONS
%----------------------------------------------------------------------------------------

\begin{rSection}{Publications}
Peer-reviewed:
\begin{etaremune}

\item {{\bf Scannell, H. A.}, L. Thompson, G. C. Johnson, J. M. Lyman, S. Riser, Subsurface evolution of recent marine heatwaves in the Northeast Pacific, \textit{Geophys. Res. Lett.}, 47, e2020GL090548, DOI: \href{https://doi.org/10.1029/2020GL090548}{10.1029/2020GL090548}. }

\item {Sen Gupta, A., M. Thomsen, J. A. Benthuysen, A. J. Hobday, E. Oliver, L. V. Alexander, M. T. Burrows, M. G. Donat, M. Feng, , N. J. Holbrook, S. Perkins-Kirkpatrick, P. J. Moore, R. R. Rodrigues, {\bf H. A. Scannell}, A. S. Taschetto, C. C. Ummenhofer, T. Wernberg, and D. Smale, Drivers and impacts of the most extreme marine heatwaves events, \textit{Sci. Rep.}, 10, 19359, DOI: \href{https://doi.org/10.1038/s41598-020-75445-3}{10.1038/s41598-020-75445-3}.}

\item {Holbrook, N. J., A. Sen Gupta, E. C. J. Oliver, A. J. Hobday, J. A. Benthuysen, {\bf H. A. Scannell}, D. A. Smale, and T. Wernberg (2020), Keeping pace with marine heatwaves as oceans warm, \textit{Nat. Rev. Earth Environ.}, 1, 482–493, DOI: \href{https://doi.org/10.1038/s43017-020-0068-4}{10.1038/s43017-020-0068-4}.}

\item 
{Holbrook, N. J., {\bf H. A. Scannell}, A. Sen Gupta, J. A. Benthuysen, M. Feng, E. C. J. Oliver, L. V. Alexander, M. T. Burrows, M. G. Donat, A. J. Hobday, P. J. Moore, S. E. Perkins-Kirkpatrick, D. A. Smale, S. C. Straub, and T. Wernberg (2019), A global assessment of marine heatwaves and their drivers, \textit{Nat. Commun.}, 10, 2624, DOI: \href{https://doi.org/10.1038/s41467-019-10206-z}{10.1038/s41467-019-10206-z}.}

\item 
{Smale, D. A., T. Wernberg, E. C. J. Oliver, M. Thomsen, B. P. Harvey, S. C. Straub, M. T. Burrows, L. V. Alexander, J. A. Benthuysen, M. G. Donat, M. Feng, A. J. Hobday, N. J. Holbrook, S. E. Perkins-Kirkpatrick, {\bf H. A. Scannell}\, A. Sen Gupta, B. Payne, and P. J. Moore (2019), Marine heatwaves threaten global biodiversity and the provision of ecosystem services, \textit{Nat. Clim. Change}, 9, 306–312, DOI: \href{http://dx.doi.org/10.1038/s41558-019-0412-1}{10.1038/s41558-019-0412-1}.}

\item {{\bf Scannell, H. A.}, and M. J. McPhaden (2018), Seasonal mixed layer temperature balance in the Southeastern Tropical Atlantic, \textit{J. Geophys. Res. Oceans}, 123, 5557–5570, DOI: \href{https://doi.org/10.1029/2018JC014099}{10.1029/2018JC014099}.}

\item 
{Oliver, E. C. J., M. G. Donat, M. T. Burrows, P. J. Moore, D. A. Smale, L. V. Alexander, J. Benthuysen, M. Feng, A. Sen Gupta, A. J. Hobday, N. J. Holbrook, S. E. Perkins-Kirkpatrick, {\bf H. A. Scannell}, S. C. Straub, and T. Wernberg (2018), Longer and more frequent marine heatwaves over the past century, \textit{Nat. Commun.}, 9, 1324, DOI: \href{https://doi.org/10.1038/s41467-018-03732-9}{10.1038/s41467-018-03732-9}.}

\item 
{Pershing, A. J., M. A. Alexander, C. M. Hernandez, L. A. Kerr, A. Le Bris, K. E. Mills, J. A. Nye, N. R. Record, {\bf H. A. Scannell}, J. D. Scott, G. D. Sherwood, and A. C. Thomas (2016), Response to Comments on “Slow adaptation in the face of rapid warming leads to collapse of the Gulf of Maine cod fishery”, \textit{Science}, 352(6284), 423, DOI: \href{https://doi.org/10.1126/science.aae0463}{10.1126/science.aae0463}.}

\item 
{{\bf Scannell, H. A.}, A. J. Pershing, M. A. Alexander, A. C. Thomas, and K. E. Mills (2016), Frequency of 
marine heatwaves in the North Atlantic and North Pacific since 1950, \textit{Geophys. Res. Lett.}, 43, DOI: \href{https://doi.org/10.1002/2015GL067308}{10.1002/2015GL067308}.}

\item 
{Hobday, A. J., L. V. Alexander, S. E. Perkins, D. A. Smale, S. C. Straub, J. Benthuysen, M. T.Burrows, M. G. Donat, M. Feng, N. J. Holbrook, P. J. Moore, E. C. J. Oliver, {\bf H. A. Scannell}, A. Sen Gupta and T. Wernberg (2016), A hierarchical approach to defining marine heatwaves, \textit{Prog. Oceanogr.}, 141: 227–238, DOI: \href{https://doi.org/10.1016/j.pocean.2015.12.014}{10.1016/j.pocean.2015.12.014}.}
    
\item 
{Pershing, A. J., M. A. Alexander, C. M. Hernandez, L. A. Kerr, A. Le Bris, K. E. Mills, J. A. Nye, N. R. Record, {\bf H. A. Scannell}, J. D. Scott, G. D. Sherwood, and A. C. Thomas (2015), Slow adaptation in the face of rapid warming leads to collapse of the Gulf of Maine cod fishery, \textit{Science}, 350(6262), 809–812, DOI: \href{https://doi.org/10.1126/science.aac9819}{10.1126/science.aac9819}.}

\end{etaremune}
\newpage

Conference Proceedings:
\begin{etaremune}
\item 
{{\bf Scannell, H. A.}, C. Fraley, N. Mannheimer, S. Battersby, and L. Thompson, Predicting marine heatwaves using global climate models with cluster based long short-term memory, 36\textsuperscript{th} International Conference on Machine Learning (ICML), Climate Change: How Can AI Help, June 14, 2019. \href{https://www.climatechange.ai/papers/icml2019/38}{[Abstract]}}
\end{etaremune}

Reports:
\begin{etaremune}
\item

{Crosman, K., L. Johnson, E. Petrou, and {\bf Scannell, H. A.}, Safeguarding Pacific Northwest Fisheries from a Warming Climate. \textit{The New York Times}, International Edition, August 2, 2017. \href{https://apru.org/wp-content/uploads/2018/12/NYT-APRU-Case-winners-booklet-571b.pdf}{[PDF]}}
\end{etaremune}

\end{rSection}


%----------------------------------------------------------------------------------------
%	HONORS AND AWARDS
%----------------------------------------------------------------------------------------

\begin{rSection}{Honors \& Awards}

National Center for Atmospheric Research Advanced Study Program Graduate Fellowship \hfill {\em 2020} \\
Integral Consulting Inc. Environmental Big Data Research Award
\hfill {\em 2018} \\
\textit{The New York Times} Asia-Pacific Case Competition First Place
\hfill {\em 2017} \\
University of Washington Program on Climate Change Graduate Fellowship
\hfill {\em 2015} \\
National Science Foundation East Asia and Pacific Summer Institute Fellowship
\hfill {\em 2014}
\end{rSection}


%----------------------------------------------------------------------------------------
%	SERVICE AND OUTREACH
%----------------------------------------------------------------------------------------

\begin{rSection}{Service \& Outreach}
Co-Chair \\ \indent
\hspace{.7 cm} Program on Climate Change Summer Institute on Climate Extremes, Marine Heatwaves \hfill {\em 2020} \\ \indent
\hspace{.7 cm} Ocean Sciences Meeting, Session on Marine Heatwaves \& Ocean Biogeochemical Extreme  \hfill {\em 2020} \\ \indent 
\hspace{.7 cm} University of Washington Graduate Climate Conference \hfill {\em 2016} \\ \indent
Advisor \\ \indent
\hspace{.7 cm} Organizing Committee for the Program on Climate Change Spring Symposium \hfill {\em 2017 – 2018} \\ \indent 
\hspace{.7 cm} Graduate Student Steering Committee for the Program on Climate Change \hfill {\em 2016 – 2018} \\ \indent 
Referee \\ \indent
\hspace{.7 cm} SciPy Conference Program Committee (3 subcommittees): \hfill {\em 2021} \\ \indent
\hspace{1.4 cm}{Data Visualization and Image Processing} \\ \indent
\hspace{1.4 cm}{Earth, Ocean, Geo, \& Atmospheric Science} \\ \indent
\hspace{1.4 cm}{Scientific Applications of Machine Learning \& Data Science} \\ \indent 
\hspace{.7 cm} NeurIPS Climate Change AI Workshop Program Committee  \hfill {\em 2019} \\ \indent
\hspace{.7 cm} Academic Journals: \hfill {\em 2017 – Present}\\ \indent
\hspace{1.4 cm}{\em Geophysical Research Letters}  \\ \indent
\hspace{1.4 cm}{\em Journal of Climate} \\ \indent
\hspace{1.4 cm}{\em Journal of Geophysical Research-Oceans}  \\ \indent 
\hspace{1.4 cm}{\em Nature Communications Earth \& Environment} \\ \indent
Volunteer \\ \indent
\hspace{.7 cm} Contributing Author, oceanbites.org \hfill {\em 2014 – 2015}  
\end{rSection}

%----------------------------------------------------------------------------------------
%	TECHNICAL STRENGTHS SECTION
%----------------------------------------------------------------------------------------

\begin{rSection}{Technical Strengths}

\begin{tabular}{ @{} >{\bfseries}l @{\hspace{6ex}} l }
Computer Languages & python, MATLAB, shell scripting \\
Data \& Databases & CESM, QG-models, NetCDF, NCO, Zarr \\
Tools & git, \LaTeX \\
Platforms & Azure, AWS, GCP, HPC, Linux \\
Core Developer & \href{https://github.com/hscannell/ocetrac/tree/master/}{ocetrac}


\end{tabular}

\end{rSection}

\newpage

%----------------------------------------------------------------------------------------
%	EXAMPLE SECTION
%----------------------------------------------------------------------------------------

\begin{rSection}{Teaching} 

{\bf Graduate Teaching Assistant} \\ 
University of Washington, School of Oceanography \\ \indent
\hspace{.7 cm} OCEAN 201: Introduction to Oceanography, 2018, 2019 \\ \indent
\hspace{.7 cm} OCEAN 285/286: Physics Across Oceanography: Fluid Mechanics and Waves, 2018 \\ \indent
\hspace{.7 cm} OCEAN 320: Coastal Oceanography, 2018 \\ \indent
\hspace{.7 cm} OCEAN 215: Methods of Oceanographic Data Analysis, 2016 \\ \indent
University of Maine, School of Marine Science \\ \indent
\hspace{.7 cm} SMS 420: Oceans and Climate Change, 2014 \\ \indent
\hspace{.7 cm} SMS 204: Integrative Marine Science II: Physics and Chemistry of Marine Systems, 2013 \\ \indent
\end{rSection}

%----------------------------------------------------------------------------------------


%----------------------------------------------------------------------------------------
%	EXAMPLE SECTION
%----------------------------------------------------------------------------------------

\begin{rSection}{Presentations}
Talks:
\begin{etaremune}

\item Scannell, H. A., 2020. West Coast Marine Heatwaves, Olympic Coast National Marine Sanctuary's Sanctuary Advisory Council meeting, remote, 25 September.

\item Scannell, H. A., 2020. Defining and Characterizing Marine Heatwaves for Prediction, US CLIVAR Predictability, Predictions, and Applications Interface  Panel Summer Meeting, remote, 21 July.

\item Scannell, H. A., S. C. Riser, L. Thompson, and G. Johnson, 2019. The 2019 reappearance of the Northeast Pacific marine heatwave, Ocean Sciences Meeting, San Diego, WA, 21 February.

\item Scannell, H. A., 2019. Integrating machine learning with traditional approaches in ocean science, Gulf of Maine Research Institute, Portland, ME, 25, November.

\item Scannell, H. A., S. C. Riser, L. Thompson, and G. Johnson, 2019. The 2019 reappearance of the Northeast Pacific marine heatwave, Physical Oceanography Seminar, School of Oceanography, University of Washington, Seattle, WA, 13 November.

\item Scannell, H. A., 2019. Rising Toll of Marine Heatwaves, Water \& Salmon Committee of the Washington State Sierra Club, Seattle, WA, 15 April.

\item Scannell, H. A., \& co-authors, 2019. Marine heatwaves threaten global biodiversity and the provision of ecosystem services, NOAA-Northwest Fisheries Science Center, Seattle, WA, 13 March.

\item Scannell, H. A., 2018. Seasonal mixed layer heat budget in the Southeast Tropical Atlantic, Ocean Sciences Meeting, Portland, Oregon.

\item Scannell, H. A., 2017. Mechanisms controlling the seasonal mixed layer heat budget in the southeast Tropical Atlantic, Program on Climate Change Spring Symposium, University of Washington, Seattle, WA, 8 April.

\item Scannell, H. A., 2017. Taking a holistic view of marine heatwaves globally, NOAA Alaskan Fisheries Science Center, Seattle, WA.

\item Scannell, H. A., 2017. Marine Heatwaves: Emerging climate phenomena, Sound Waters University, Langley, WA.

\item Scannell, H. A., 2016. Marine Heatwaves: Emerging climate phenomena, Program on Climate Change Graduate Student Seminar, University of Washington, Seattle, WA.

\item Scannell, H. A. Climatic influences on extra-tropical marine heatwaves, Commonwealth Scientific and Industrial Research Organization, Oceans and Atmosphere Flagship Seminar Series, Perth, Western Australia. 2015.


\newpage

\item {Scannell, H. A.. The ocean heatwave phenomenon and the climatic mechanisms at play. Climate Change Research Center Seminar, University of New South Wales, Sydney, Australia. August 2014.}
\end{etaremune}


Posters:
\begin{etaremune}

\item Scannell, H. A., L. Thompson, W. Cheng, and E. Maroon, 2019. Characterization of marine heatwaves in the CESM Large Ensemble, US CLIVAR Large Ensembles Workshop, NCAR, Boulder, CO.

\item Scannell, H. A., and M. J. McPhaden, 2017. Seasonal mixed layer heat budget in the southeast tropical Atlantic, American Meteorological Society Annual Meeting, Seattle, WA, 25 January, Abstract \#308397.

\item Scannell, H. A., and M. J. McPhaden, 2016. Seasonal mixed layer heat budget in the southeast tropical Atlantic, AGU Fall Meeting, San Francisco, CA, 13 December, Abstract \#152486.

\item Scannell, H. A., and M. J. McPhaden, 2016. Seasonal mixed layer heat budget in the southeast tropical Atlantic, Graduate Climate Conference, Pack Forest, WA, 29 October.

\item Scannell, H. A., M. H. England, and A. Sen Gupta, 2014. Climatic influences on Indian and Pacific Ocean heatwaves, AGU Fall Meeting, San Francisco, CA, 18 December, Abstract \#OS43A-1262.

\item Scannell, H. A., M. H. England, and A. Sen Gupta, 2014. Quasi-decadal variability of ocean heatwaves in the Southern Hemisphere extra-tropics, Graduate Climate Conference, Pack Forest, WA, 1 November.

\item Scannell, H. A., A. J. Pershing, and K. E. Mills, 2014. Frequency of ocean heatwaves occurring in the Atlantic and Pacific Oceans, Ocean Sciences Meeting, Honolulu, HI, 24 February, Abstract \#1389.

\item Scannell, H. A., A. J. Pershing, and K. E. Mills, 2013. Likelihood of an ocean heatwave in the northwest Atlantic Ocean, Regional Association for Research on the Gulf of Maine, Portsmouth, NH, 8 October.

\end{etaremune}


\end{rSection}

%----------------------------------------------------------------------------------------




%----------------------------------------------------------------------------------------
%	EXAMPLE SECTION
%----------------------------------------------------------------------------------------

\begin{rSection}{Media Spotlights and Interviews}

\href{https://www.npr.org/2019/09/25/755859707/earths-oceans-are-getting-hotter-and-higher-and-it-s-accelerating}{\textbf{NPR}}, Earth's Oceans Are Getting Hotter And Higher, And It's Accelerating, 9/25/19

\href{https://www.washingtonpost.com/weather/2019/09/21/blob-is-surging-back-pacific-leading-fears-mass-die-offs-marine-life-unusual-weather-patterns/}{\textbf{The Washington Post}}, The ‘Blob’ is surging back in the Pacific, leading to fears of mass die-offs of marine life and unusual weather patterns, 9/21/19

\href{https://insideclimatenews.org/news/17092019/marine-heat-wave-climate-change-pacific-coast-fish-wildlife-california-wildfires-hurricanes-hawaii}{\textbf{InsideClimate News}}, A Marine Heat Wave Intensifies, with Risks for Wildlife, Hurricanes and California Wildfires, 9/18/19

\href{https://www.forbes.com/sites/allenelizabeth/2019/09/05/another-warm-blob-is-forming-in-the-pacific-ocean/#6f979ec814af}{\textbf{Forbes}}, Another ‘Warm Blob’ Is Forming In The Pacific Ocean, 9/5/19

\href{https://insideclimatenews.org/news/06022019/climate-change-scientists-ecosystem-disappearing-mountain-glaciers-ice-forests-oceans}{\textbf{InsideClimate News}}, 5 Science Teams Racing Climate Change as the Ecosystems They Study Disappear, 2/6/19

\href{https://eos.org/features/why-is-the-gulf-of-maine-warming-faster-than-99-of-the-ocean}{\textbf{EOS Earth \& Space Science News}}, Why Is the Gulf of Maine Warming Faster Than 99\% of the Ocean? 11/12/18

\href{https://www.carbonbrief.org/restricting-global-warming-to-1-5c-would-halve-risk-of-marine-heatwaves}{\textbf{Carbon Brief}}, Restricting global warming to 1.5\textsuperscript{o}C would ‘halve’ risk of marine heatwaves, 8/15/18

\href{https://www.washington.edu/news/2017/06/26/the-new-york-times-recognizes-uw-student-policy-recommendations/}{\textbf{UW Today}}, The New York Times recognizes UW student policy recommendations, 6/26/17

\href{http://www.dailyuw.com/science/article_e6abdc30-0cf3-11e6-a211-7bcf0ff942f1.html#user-comment-area}{\textbf{The Daily}}, Understanding the blob and 65 years of hot water, 4/28/16

\href{https://www.wunderground.com/blog/JeffMasters/the-north-atlantic-blob-a-marine-cold-wave-that-wont-go-away.html}{\textbf{Weather Underground}}, The North Atlantic Blob: A Marine Cold Wave That Won’t Go Away, 4/8/16

\href{http://mynorthwest.com/11/2943491/Pacific-Oceans-marine-heatwaves-likely-to-become-more-frequent-intense}{\textbf{MyNorthwest}}, Pacific Ocean’s ‘marine heatwaves’ likely to become more frequent, intense, 4/3/16

\href{http://www.king5.com/tech/science/environment/turns-out-the-blob-isnt-alone/112940327}{\textbf{KING 5 News}}, Global ‘blobs’ getting more extreme, 4/1/16

\href{http://www.washington.edu/news/2016/03/30/tracking-marine-heatwaves-since-1950-and-how-the-blob-stacks-up/}{\textbf{UW Today}}, Tracking ‘marine heatwaves’ since 1950 – and how the ‘blob’ stacks up, 3/30/16

\href{http://www.hakaimagazine.com/article-short/revenge-blob}{\textbf{Hakai Magazine}}, Revenge of the Blob, 3/2/16 

\href{https://www.deutschlandfunk.de/klimaforschung-hitzewellen-auf-see.676.de.html?dram\%3Aarticle_id=346758}{\textbf{German National Public Radio}}, Heatwaves at Sea, 2/25/16

\href{http://www.techtimes.com/articles/101190/20151030/climate-change-blamed-for-collapse-of-iconic-new-england-cod-populations.htm}{\textbf{Tech Times}}, Climate change blamed for collapse of iconic New England Cod Populations, 10/30/15

\href{http://www.npr.org/sections/thesalt/2015/10/29/451942641/why-is-it-so-hard-to-save-gulf-of-maine-cod-theyre-in-hot-water}{\textbf{NPR}}, Why is it so hard to save Gulf of Maine Cod? They’re in hot water, 10/30/15

\href{http://www.nytimes.com/2015/10/30/science/cods-continuing-decline-traced-to-warming-gulf-of-maine-waters.html?_r=0}{\textbf{The New York Times}}, Cod’s Continuing Decline Linked to Warming Gulf of Maine Waters, 10/29/15

\href{https://www.bostonglobe.com/metro/2015/10/29/climate-change-seen-contributing-collapse-cod-stocks-off-new-england/WIMPUHnYWBwNBDJ0QpcqWO/story.html}{\textbf{Boston Globe}}, Climate change hurting N.E. cod population, study says, 10/29/15

\href{http://www.sciencemag.org/news/2015/10/collapse-new-england-s-iconic-cod-tied-climate-change}{\textbf{Science Magazine}}, Collapse of New England’s iconic cod tied to climate change, 10/29/15

\href{https://www.sciencedaily.com/releases/2015/10/151029150250.htm}{\textbf{Science Daily}}, Warming waters a major factor in the collapse of New England cod, 10/29/15

\href{http://www.wired.com/2015/10/climate-change-is-wicked-bad-for-new-englands-cod-fishery/}{\textbf{Wired}}, Climate change is wicked bad for New England’s cod, 10/29/15

\href{http://www.smithsonianmag.com/science-nature/why-smarter-fishing-practices-arent-saving-maine-cod-collapse-180957096/?no-ist}{\textbf{Smithsonian}}, Why smarter fishing practices aren’t saving Maine cod from collapse, 10/29/15

\href{http://www.eurekalert.org/pub_releases/2015-10/aaft-nec102615.php}{\textbf{EurekAlert}}, New England cod collapse linked to warming waters, 10/29/15

\href{http://www.pressherald.com/2015/10/29/study-climate-change-in-gulf-of-maine-is-responsible-for-cods-failure-to-recover/}{\textbf{Portland Press Herald}}, Climate change in Gulf of Maine is responsible for cod’s failure to recover, 10/29/15

\href{http://www.cbsnews.com/news/a-warmer-atlantic-is-helping-kill-off-the-cod/}{\textbf{CBS News}}, Warmer Atlantic a big factor in cod kill-off, study shows, 10/29/15

\href{https://www.washingtonpost.com/news/energy-environment/wp/2015/10/29/climate-change-is-doing-some-strange-things-to-the-waters-off-new-england/}{\textbf{The Washington Post}}, Climate change is doing some very strange things to the waters off New England, 10/29/15

\href{http://theplate.nationalgeographic.com/2015/10/29/warm-water-may-spell-the-end-of-new-englands-iconic-cod/}{\textbf{National Geographic}}, Warm water may spell the end to New England’s iconic cod, 10/29/15

\href{http://www.carbonbrief.org/the-contrasting-fortunes-of-atlantic-cod-in-warming-oceans}{\textbf{Carbon Brief}}, The contrasting fortunes of Atlantic cod in warming oceans, 10/29/15

\href{http://www.aaas.org/news/science-warming-waters-contribute-collapse-atlantic-cod}{\textbf{AAAS}}, Warming waters contribute to collapse of Atlantic Cod, 10/28/15

\endgroup

\end{rSection}

%----------------------------------------------------------------------------------------
\end{document}
